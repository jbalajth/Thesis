\chapter{Detector Response to $^{14}$C Beta Decay}

\section{The Theoretical $^{14}$C Beta-Spectrum}
Carbon-14 decays to the ground state of nitrogen-14 by way of an allowed Gamow-Teller transition. This decay has a Q-value of 156 keV and a half life of 5730 years In general, the theoretical spectrum for this decay takes the form:

The theoretical spectrum of the carbon-14 beta decay to nitrogen-14 has been of interest in the context neutrino mass experiments\cite{C14_Wietfeldt}, liquid scintillator experiments\cite{C14_Borexino, C14_Bergeron}, as well as theoretical nuclear physics\cite{C14_Kuzminov,C14_Genz,C14_Garcia}. In all of these cases, it is important to be able to precisely predict and measure the shape of the $^{14}$C beta spectrum. There is, however, some continuing disagreement among both theory and experiment. This disagreement disagreement lies in the calculation of the induced current shape factor


\subsection{The Kinematic Spectrum}
\subsection{The Fermi Function}
\subsection{The Shape Function}



\section{Measurement of the Shape of the $^{14}$C Beta-Spectrum}


\section{Measurement of Light and Charge Yields}


\section{Measurement of Recombination Fluctuations from $^{14}$C}